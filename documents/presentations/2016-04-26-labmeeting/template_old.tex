% $Header: /Users/joseph/Documents/LaTeX/beamer/solutions/conference-talks/conference-ornate-20min.en.tex,v 90e850259b8b 2007/01/28 20:48:30 tantau $

\documentclass{beamer}

% This file is a solution template for:

% - Talk at a conference/colloquium.
% - Talk length is about 20min.
% - Style is ornate.



% Copyright 2004 by Till Tantau <tantau@users.sourceforge.net>.
%
% In principle, this file can be redistributed and/or modified under
% the terms of the GNU Public License, version 2.
%
% However, this file is supposed to be a template to be modified
% for your own needs. For this reason, if you use this file as a
% template and not specifically distribute it as part of a another
% package/program, I grant the extra permission to freely copy and
% modify this file as you see fit and even to delete this copyright
% notice. 


\mode<presentation>
{
  \usetheme{default}
  \usecolortheme{beaver}
  %\usecolortheme{lily}

}


\usepackage[english]{babel}
% or whatever

\usepackage[latin1]{inputenc}
% or whatever

\usepackage{times}
\usepackage[T1]{fontenc}
\usepackage{graphics}
\usepackage{caption}
% Or whatever. Note that the encoding and the font should match. If T1
% does not look nice, try deleting the line with the fontenc.


\title[Some title] % (optional, use only with long paper titles)
{13.5.2 Relational Processing Theory: Refinement of the Declareative Memory Theory}

\author % (optional, use only with lots of authors)
{Seong-Beom Park  \and Daniel M. Bjoernstad \and Sumin Lee}
% - Give the names in the same order as the appear in the paper.
% - Use the \inst{?} command only if the authors have different
%   affiliation.




% If you have a file called "university-logo-filename.xxx", where xxx
% is a graphic format that can be processed by latex or pdflatex,
% resp., then you can add a logo as follows:

% \pgfdeclareimage[height=0.5cm]{university-logo}{university-logo-filename}
% \logo{\pgfuseimage{university-logo}}



% Delete this, if you do not want the table of contents to pop up at
% the beginning of each subsection:
%
%\AtBeginSubsection[]
%{
  %\begin{frame}<beamer>{Outline}
  %  \tableofcontents[currentsection,currentsubsection]
  %\end{frame}
%}


% If you wish to uncover everything in a step-wise fashion, uncomment
% the following command: 

%\beamerdefaultoverlayspecification{<+->}


\begin{document}

\begin{frame}
  \titlepage
\end{frame}

%\begin{frame}{Outline}
%  \tableofcontents
  % You might wish to add the option [pausesections]
%\end{frame}


% Structuring a talk is a difficult task and the following structure
% may not be suitable. Here are some rules that apply for this
% solution: 

% - Exactly two or three sections (other than the summary).
% - At *most* three subsections per section.
% - Talk about 30s to 2min per frame. So there should be between about
%   15 and 30 frames, all told.

% - A conference audience is likely to know very little of what you
%   are going to talk about. So *simplify*!
% - In a 20min talk, getting the main ideas across is hard
%   enough. Leave out details, even if it means being less precise than
%   you think necessary.
% - If you omit details that are vital to the proof/implementation,
%   just say so once. Everybody will be happy with that.


\begin{frame}{What is Predictable Ambiquity}{}
		\centering
		\includegraphics[height=6cm]{pic1.png}
\end{frame}


\begin{frame}{Example(Exclusive OR Problem)}
		\begin{minipage}{\linewidth}
		\centering
		\includegraphics[width=6cm]{Start1.png}
			
		\end{minipage}
		\centering
		\begin{minipage}{\linewidth}
				\centerline{Is this kind of rule possible? Yes! But how?}
		\end{minipage}
\end{frame}

\begin{frame}{Three Theories}
		\centerline{To explain predictable ambiguity three theories appeared.}
		\begin{enumerate}
			\item
					Configural Association Theory
			\item
					\alert{Relational Processing Theory}
			\item
					Contextual encoding and retrival 
					
			
		\end{enumerate}
\end{frame}

\begin{frame}{Relational Processing Theory}

		\begin{minipage}{\linewidth}
		\centering
		\includegraphics[width=10cm]{Start2.png}
			
		\end{minipage}

		\begin{minipage}{\linewidth}
		Relation:
		\begin{itemize}
			\item
				Causality: Cause-effect
			\item
				Physical relation: Near/far away
			\item
				Abstract relation: (Next slide)
			
		\end{itemize}

		\end{minipage}
	
\end{frame}

\begin{frame}{Abstract Relation}
		\begin{minipage}{\linewidth}
		\centering
		\includegraphics[width=7cm]{Start3.png}
		\end{minipage}
\end{frame}

\begin{frame}{Flexibility}
		\centering
		\includegraphics[height=5.5cm]{Start4.png}
	
\end{frame}

\begin{frame}{Neural Substrate for Relational Processing Theory}
		\begin{minipage}{0.4\linewidth}
				\centering
				\includegraphics[width=\textwidth]{hippocampus.jpg}
		\end{minipage}
		\begin{minipage}{0.5\linewidth}
\hfill
		\begin{itemize}
			\item
				Technique to find out: Lesions Experiments.
			\item
				Where to lesion: RPT builds on Declarative Memory Theory. 
			\item 
				The hippocampus is the main processor of declarative memories.
		\end{itemize}
		\end{minipage}
\end{frame}

\begin{frame}{Lesion Experiments with Rats}
		\begin{minipage}{0.4\linewidth}
				\begin{figure}[h!]
				\centering
				\includegraphics[width=\textwidth]{RatExperiment1.png}
				\caption*{Basic associations}
				\end{figure}
				\begin{figure}[h!]
				\centering
				\includegraphics[width=\textwidth]{RatExperiment2.png}
				\caption*{Inferred associations}
				\end{figure}
		\end{minipage}
		\begin{minipage}{0.5\linewidth}
		\hfill
		\begin{itemize}
			\item
				Bunsey and Eisenbaum's (1996) paired association task.
			\item
				Learning phase: Basic associations
			\item 
				Test phase: Infere the correct result from basic associations
			\item
				Transative inference task: Test the rats ability to infere.
		\end{itemize}
		\end{minipage}
		
\end{frame}

\begin{frame}{Results}
		\begin{minipage}{\linewidth}
				\centering
				\includegraphics[width=.45\textwidth]{RatResult1.png}
				\centering
				\includegraphics[width=.45\textwidth]{RatResult2.png}
		\end{minipage}

		\begin{minipage}{\linewidth}
		\begin{itemize}
			\item
				Selective imparement in transitive inference tasks.
		\end{itemize}
		\end{minipage}
	
\end{frame}

\begin{frame}{Support for Relational Processing Theory}
		\begin{itemize}
	\item
				Experiments on rats support RPT is hippocampus dependent.
		\item
				Experiments on monkeys supports the same.
		\item
				Mixed support in humans.
		\item 
				Mixed support in other species (birds).
		\end{itemize}
\end{frame}


\begin{frame}{Social Transmission of Food Preference}
\begin{minipage}{0.3\linewidth}
		\centering
			\includegraphics[width=\textwidth]{RatSocial.png}
\end{minipage}
\hfill
\begin{minipage}{0.65\linewidth}
		\begin{itemize}
		\item
		\alert{Exchange of Information:} Information about novel food is acquired
		from the breath of the demonstrator rather than actual consuption. 
		\item
		\alert{Prefernce Test:}Rat can relate the learned information to the
		odor of actual food.
		\end{itemize}
\end{minipage}

\begin{minipage}{\linewidth}
\vspace{1cm}
\begin{itemize}
		\item
"Relational processing enables flexible access to information."
\end{itemize}
\end{minipage}

\end{frame}


\begin{frame}{Social Transmission of Food Preference}
			\includegraphics[width=\textwidth]{RatExperiment3.png}
		\begin{itemize}
		\item
				Fast forgetting after 2 days
			
		\end{itemize}
\end{frame}

\begin{frame}{Social Transmission of Food Preference}
			\includegraphics[width=\textwidth]{RatExperiment4.png}
		\begin{itemize}
		\item
				Fast forgetting after 2 days
		\item
				Consolidation gradient is observed
		\item
				Hippocampus is important for consolidation in relational processing.
			
		\end{itemize}
\end{frame}

\begin{frame}{Functional dissociation between hippocampus and neocortical regions in MTL}
		\begin{columns}
			
		\begin{column}{0.48\linewidth}
				\alert{Squire's Declarative Memory Theory:}
			\begin{itemize}
				\item
					There is no functional distinctions in MTL
				
			\end{itemize}
		\end{column}
		\hfill

		\begin{column}{0.48\linewidth}
		\alert{Relational Processing Theory:}
			\begin{itemize}
				\item
					Neocortical structures in MTL: Storage of individual items.
				\item
					Hippocampus: Relational processor.
				
			\end{itemize}
			
		\end{column}
		\end{columns}

		\vspace{1cm}
		The limit of this theory:

		Only one type of information -> Olfactory stimulus
\end{frame}

\section*{Summary}

\begin{frame}{Summary}
		\begin{columns}
			
		\begin{column}{0.3\linewidth}
		\begin{figure}[h!]
		\centering
		\includegraphics[height=1cm]{RatExperiment1.png}
		\caption*{Basic associations}
		\end{figure}
		\begin{figure}[h!]
		\centering
		\includegraphics[height=1cm]{RatExperiment2.png}
		\caption*{Inferred associations}
		\end{figure}
		\end{column}
		\hfill

		\begin{column}{0.6\linewidth}
	Relational Processing Theory:
  % Keep the summary *very short*.
  \begin{itemize}
	\item
			Store basic associations.
  \item
		  Use basic associations between representations to infere new "knowledge".
  \item
		  Flexible recall in different circumstances.
  \end{itemize}
  
  % The following outlook is optional.
  RPT vs. Configurational Theory:
  \begin{itemize}
  \item
		  Different from configurational by using relations between representations.
	\item
			Configural theory uses the representations serially.
  \end{itemize}
		\end{column}
		\end{columns}


\end{frame}

\end{document}


