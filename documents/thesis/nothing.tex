% Overview {{{ %
\chapter*{Overview}

Abstract (1/10):
\newline
Summary of the summary. A short introduction with results stated clearly.
\newline

Introduction (1/10):
\newline
Introduce the topic, the problem and how the problem is being solved. 
The target audience are other master students. 
% Keywords: Action potential shape, cell classification, electrodes.
\newline 

Theory:
\begin{itemize}
    \setlength\itemsep{0pt}
    \item Basic information about neurons and the cell membrane. (7/10)
    \item Basic information about different types of neurons, how they serve
        different purposes and have different functions. (0/10)
    \item Turning the neuron into a electronic circuit, explaining ion pumps 
        and channels. (5/10)
    \item Explanation of action potentials and the generation of
        action potentials with a the hudgekin and huxley model. (1/10)
    \item Explanation of the principle of compartmental models with diagrams.
        There are many kinds. Not too long. (0/10)
    \item Explanation of electrodes, how they are used,
        what they measure, how tetrodes work. (1/10)
    \item Mentioning that neurons measured from the same electrode must
        be seperated. Cocktail party problem, source seperation. (0/10)
    \item Explanation of extracellular potential, the physics behind the
        problem, current sum to zero, how it can be calculated, etc.(4/10)
    \item Discussion of the difference between intracellular and extracellular 
        spike shape. Some say the extracellular spike is the derivative. (0/10)
    \item Explanation of Neuron and LFPy. How they work, the principle 
        behind them, what they are intended to calculate. (2/10)
    \item Explanation of the current state of cell classification.
        Why is it important, how is it done. Mentioning most influential 
        work from early to current work. (0/10)
    \item Explanation of the Blue Brain cell database, how it was made, 
        why is it useful, what were the focus of the models, models are public. (0/10)
\end{itemize}
\noindent 
Note to self, mention the most influential work that has been done on all
topics mentioned in the theory. I should show I know all the basic 
important work that has been done in these fields. 
\newline

Methods
\newline
Everything here is work which I have done myself. Show what I have done, make sure
it understood as a lot.
\begin{itemize}
    \item Explanation of the basis of differentiating spikes, how does
        one measure how spikes are different. Meantion spike width and 
        ampltiude. (0/10)
    \item Explain different definitions of spike width and amplitude measurements. (1/10)
    \item Detailed explanation about the simulation environment. 
        What does LFPyUtil solve and how does it solve it. (3/10)
    \item Showing a minimum working example of LFPyUtil, show what it makes easier. (0/10)
    \item Detailed explanation of each simulation, what are the parameters. (3/10)
    \item How did I use the BlueBrain models, which models are used, why are they used. (0/10)
\end{itemize}

Results:
\newline
State the results in such a way they clearly show what I want to show, but
does not "jump to conclusions". State the results without bias.
Include figures with text, at a glance the figures will be read seen and read
first. 
\begin{itemize}
    \item Detailed explanation of the replication of Pettersen and Einevoll, show 
        that the simulation environment can be trusted. (6/10)
    \item Explain any deviations from Pettersen and Einevoll in the replication. (5/10)
    \item Create a conclusion of the results from Pettersen and Einevoll. (0/10)
    \item Show which definition is best for differentiating spikes from different
        kinds of neurons. (0/10)
    \item Show that interneurons and pyramidal neurons can be classified. (0/10)
    \item Explain why spikes look differnt, what are the physical processes that 
        does this? (0/10)
\end{itemize}

Discussion:
\newline
The discussion is important, make it of high quality and maybe long. 
\begin{itemize}
    \item Using spike width have already been used for differentiate neurons, 
        show how current results backs up this statement. (0/10)
    \item Research has shown that thin spikes can also come from measuring near
        axons of pyramidal cells. (0/10)
    \item Argue that the models are relateable to real spikes. (0/10)
    \item The analysis framework can be used for future cell models. (0/10)
\end{itemize}

Note to self, when trying to explain a method first show the problem clearly
then propose the solution to the problem. Engage the reader by showing the 
problem in such a way they can become curious for a solution. 

Spike shape is defined by the internal mecanics and the morphology.
How much is defined by each.
Can areas be defined, yes.
Assume each group are the same, because it is defined as the same.
They still show diversity.
Show their areas and their overlap.


% }}} Overview %
